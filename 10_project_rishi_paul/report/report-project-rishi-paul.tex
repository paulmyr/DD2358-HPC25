\documentclass[a4paper,12pt]{article}
\usepackage[utf8]{inputenc}
\usepackage[a4paper,
            bindingoffset=0.2in,
            left=1in,
            right=1in,
            top=1in,
            bottom=1in,
            footskip=.25in]{geometry}

%###############################################################################

%\input{~/layout/global_layout}


%###############################################################################

% packages begin

\usepackage[
  backend=biber,
  sortcites=true,
  style=alphabetic,
  eprint=true,
  backref=true
]{biblatex}
\addbibresource{bibliographie.bib}
\usepackage[acronym]{glossaries}

\usepackage{euscript}[mathcal]
% e.g. \mathcal{A} for fancy letters in mathmode
\usepackage{amsmath,amssymb,amstext,amsthm}

\usepackage{mdframed}
\newmdtheoremenv[nobreak=true]{problem}{Problem}[subsection]
\newmdtheoremenv[nobreak=true]{claim}{Claim}[subsection]
\newtheorem{definition}{Definition}[subsection]
\newtheorem{lemma}{Lemma}[claim]
\newtheorem{plemma}{Lemma}[problem]

\usepackage{mathtools}
\DeclarePairedDelimiter\ceil{\lceil}{\rceil}
\DeclarePairedDelimiter\floor{\lfloor}{\rfloor}

\usepackage{enumerate}
\usepackage[pdftex]{graphicx}
\usepackage{subcaption}
% 'draft' für schnelleres rendern mitübergeben -> [pdftex, draft]
% dadruch wird nicht das bild mitgerendered, sondern nur ein kasten mit bildname -> schont ressourcen

\usepackage{hyperref}

\usepackage{tikz}
\usetikzlibrary{arrows,automata,matrix,positioning,shapes}

% for adding non-formatted text to include source-code
\usepackage{listings}
\lstset{language=Python,basicstyle=\footnotesize}
% z.B.:
% \lstinputlisting{source_filename.py}
% \lstinputlisting[lanugage=Python, firstline=37, lastline=45]{source_filename.py}
%
% oder
%
% \begin{lstlisting}[frame=single]
% CODE HERE
%\end{lstlisting}
\usepackage{algorithm}
\usepackage{algpseudocode}

\usepackage{wasysym}

\usepackage{titling}
\usepackage{titlesec}
\usepackage[nocheck]{fancyhdr}
\usepackage{lastpage}

\usepackage{kantlipsum}
\usepackage[colorinlistoftodos,prependcaption,textsize=tiny]{todonotes}

% packages end
%###############################################################################

\pretitle{% add some rules
  \begin{center}
    \LARGE\bfseries
} %, make the fonts bigger, make the title (only) bold
\posttitle{%
  \end{center}%
  %\vskip .75em plus .25em minus .25em% increase the vertical spacing a bit, make this particular glue stretchier
}
\predate{%
  \begin{center}
    \normalsize
}
\postdate{%
  \end{center}%
}

\titleformat*{\section}{\Large\bfseries}
\titleformat*{\subsection}{\large\bfseries}
\titleformat*{\subsubsection}{\normalsize\bfseries}

\titleformat*{\paragraph}{\Large\bfseries}
\titleformat*{\subparagraph}{\large\bfseries}

%###############################################################################
% TODO define Headers and Fotter

\pagestyle{fancy}
\fancyhf{}
% l=left, c=center, r=right; e=even_pagenumber, o=odd_pagenumber; h=header, f=footer
% example: [lh] -> left header, [lof,ref] -> fotter left when odd, right when even
%\fancyhf[lh]{}
%\fancyhf[ch]{}
%\fancyhf[rh]{}
%\fancyhf[lf]{}
\fancyhf[cf]{\footnotesize Page \thepage\ of \pageref*{LastPage}}
%\fancyhf[rf]{}
\renewcommand{\headrule}{} % removes horizontal header line

% Fotter options for first page

\fancypagestyle{firstpagestyle}{
  \renewcommand{\thedate}{\textmd{}} % removes horizontal header line
  \fancyhf{}
  \fancyhf[lh]{\ttfamily M.Sc. Computer Science\\KTH Royal Institute of Technology}
  \fancyhf[rh]{\ttfamily Period 3\\\today}
  \fancyfoot[C]{\footnotesize Page \thepage\ of \pageref*{LastPage}}
  \renewcommand{\headrule}{} % removes horizontal header line
}
%###############################################################################
% Todo: define Title

\title{
  \normalsize{DD2358 VT25 Introduction to}\\
  \normalsize{High Performance Computing}\\
  \large{Project Report}
}
\author{
  \small Rishi Vijayvargiya\\[-0.75ex]
%  \footnotesize\texttt{MN: }\\[-1ex]
  \scriptsize\texttt{rishiv@kth.se}
  \and
  \small Paul Mayer\\[-0.75ex]
%  \footnotesize\texttt{MN: }\\[-1ex]
  \scriptsize\texttt{pmayer@kth.se}
}
\date{}

%###############################################################################
% define Commands

\newcommand{\N}{\mathbb{N}}
\newcommand{\R}{\mathbb{R}}
\newcommand{\Z}{\mathbb{Z}}
\newcommand{\I}{\mathbb{I}}

\newcommand{\E}{\mathbb{E}}
\newcommand{\Prob}{\mathbb{P}}

\renewcommand{\epsilon}{\varepsilon}

% Todo: Set Counter to Excercise Sheet Number
%\setcounter{section}{1}
%\setcounter{subsection}{1}

%###############################################################################
%###############################################################################

\begin{document}
\maketitle
\thispagestyle{firstpagestyle}

% \tableofcontents
\listoftodos

\vspace{1em}

% content begin
%

\section*{Prefix}
The original code from this project is taken from \url{https://github.com/pmocz/finitevolume-python/blob/master/finitevolume.py}.\\
We mad following changes (bugfixes) before starting with the project:

\begin{enumerate}
  \item Fixed a bug that resulted in an infinite loop when setting the \verb|plotRealTime| flag to \verb|False|.
  \item ...\todo{Add more bugs that may come up!}
\end{enumerate}

\todo[inline]{Report outline}
\section*{Preliminary}
\section{Introduction}
The Kelvin-Helmholz-Inequality is a phenomenom that arises when two fluid layers of different velocities interface with each other.
It results in characteristic patterns that can be observered, e.g. in clouds, the surface of the sun or in jupyters colorful atmosphere.
Philip Mocz created an which produces a visualization of the characteristic swirls of the K-H-inequality.

The Finite Volume method is a popular simulation technique to simulate fluids or partial differential equations that can be represented in a conservative form.
This is often the case for equations that describe physical conservation laws.
The Euler-Fluid-Equations (a simplifications of the Navier-Stokes-Equations) can be represented nebsts it's primitive description in such conservative form.
Within the algorithm, Philip Mosz makes use of both representations of the formula.
He uses the primitive form to extrapolate values in time and space, but uses the conservative representation to derive the update formula.
In some sorts, this gives us the best of both worlds.
Extrapolating within the primitive form is numerically more stable whereas the conservative representation is used to compute the update efficiently.
\todo{describe finite volume approach, maybe small image}
\todo{loose a word or two about boundary conditions and initial values}

The algorithm in principle works as follows:
For each timestep and each cell do
\begin{enumerate}
  \item Get cell central primitive variables (convert from conservative ones)
  \item Calculate timestep (What does that mean exactly)
  \item Calculate gradient of primitive vars (e.g using central differences)
  \item Extrapolate prim. vars. in time ($\Delta t / 2$)
  \item Extrapolate prim. vars. to faces ($\Delta x / 2$)
  \item Compute Fluxes along each face
  \item Update solution by adding Fluxes to conserv. vars
\end{enumerate}

Update Formula:
\begin{equation}
Q^{(n+1)} = Q^{(n)} - \Delta t \Delta x \sum_{j} \hat{F}_{ij}^{(n + \frac{1}{2})}
\end{equation}

We aim to introduce some optimization techniques that decrease the overall runtime of the computation.
To achieve this, we use techniques obtained from the lectures, specifically using cython to use precompiled c code to reduce python overhead,
introducing gpu accellerators using pytorch and investigate a distributed computation approach utilizing dask.

\section{Baseline tests}

\section{Optimizations using Pytorch}
The original implementation makes heavy use of numpy functionalities.
All data is stored as numpy arrays that are incrementally updated using the computed fluxes.
This provides an easy approach to move the computations on a GPU using pytorch.
Pytorch's computations are run using tensors --- an immutable datatype that shares resemblence with numpy arrays.
A big difference however is that tensor operations are supported to be computed using accelerators.
Due to the fact that pytorch mirrors numpys interface, we simply need to convert said numpy arrays to pytorch tensors
and tell pytorch to perform the computations using GPU compatible hardware.

To compare the performance on a wider range of grid sizes, we chose to limit the simulation to
As one can see in the results of figure [FIGURE!],
$\rightarrow$ recompute plots...

\section{Appendix:}

% content end
%###############################################################################

% TODO: bibliograpghy when needed
% \printbibliography

\end{document}
