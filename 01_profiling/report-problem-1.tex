
\documentclass[a4paper,12pt]{article}
\usepackage[utf8]{inputenc}
\usepackage[a4paper,
            bindingoffset=0.2in,
            left=1in,
            right=1in,
            top=1in,
            bottom=1in,
            footskip=.25in]{geometry}


%###############################################################################

%\input{~/layout/global_layout}


%###############################################################################

% packages begin

\usepackage[
  backend=biber,
  sortcites=true,
  style=alphabetic,
  eprint=true,
  backref=true
]{biblatex}
\addbibresource{bibliography.bib}

\usepackage{euscript}[mathcal]
% e.g. \mathcal{A} for fancy letters in mathmode
\usepackage{amsmath,amssymb,amstext,amsthm}

\usepackage{mdframed}
\newmdtheoremenv[nobreak=true]{problem}{Problem}[subsection]
\newmdtheoremenv[nobreak=true]{claim}{Claim}[subsection]
\newtheorem{definition}{Definition}[subsection]
\newtheorem{lemma}{Lemma}[claim]
\newtheorem{plemma}{Lemma}[problem]

\usepackage{mathtools}
\DeclarePairedDelimiter\ceil{\lceil}{\rceil}
\DeclarePairedDelimiter\floor{\lfloor}{\rfloor}

\usepackage{enumerate}
\usepackage[pdftex]{graphicx}
\usepackage{subcaption}
% 'draft' für schnelleres rendern mitübergeben -> [pdftex, draft]
% dadruch wird nicht das bild mitgerendered, sondern nur ein kasten mit bildname -> schont ressourcen

\usepackage{hyperref}

\usepackage{tikz}
\usetikzlibrary{arrows,automata,matrix,positioning,shapes}

% for adding non-formatted text to include source-code
\usepackage{listings}
\lstset{language=Python,basicstyle=\footnotesize}
% z.B.:
% \lstinputlisting{source_filename.py}
% \lstinputlisting[lanugage=Python, firstline=37, lastline=45]{source_filename.py}
%
% oder
%
% \begin{lstlisting}[frame=single]
% CODE HERE
%\end{lstlisting}
\usepackage{algorithm}
\usepackage{algpseudocode}

\usepackage{wasysym}

\usepackage{titling}
\usepackage{titlesec}
\usepackage[nocheck]{fancyhdr}
\usepackage{lastpage}

\usepackage{kantlipsum}
\usepackage[colorinlistoftodos,prependcaption,textsize=tiny]{todonotes}

% packages end
%###############################################################################

\pretitle{% add some rules
  \begin{center}
    \LARGE\bfseries
} %, make the fonts bigger, make the title (only) bold
\posttitle{%
  \end{center}%
  %\vskip .75em plus .25em minus .25em% increase the vertical spacing a bit, make this particular glue stretchier
}
\predate{%
  \begin{center}
    \normalsize
}
\postdate{%
  \end{center}%
}

\titleformat*{\section}{\Large\bfseries}
\titleformat*{\subsection}{\large\bfseries}
\titleformat*{\subsubsection}{\normalsize\bfseries}

\titleformat*{\paragraph}{\Large\bfseries}
\titleformat*{\subparagraph}{\large\bfseries}

%###############################################################################
% TODO define Headers and Fotter

\pagestyle{fancy}
\fancyhf{}
% l=left, c=center, r=right; e=even_pagenumber, o=odd_pagenumber; h=header, f=footer
% example: [lh] -> left header, [lof,ref] -> fotter left when odd, right when even
%\fancyhf[lh]{}
%\fancyhf[ch]{}
%\fancyhf[rh]{}
%\fancyhf[lf]{}
\fancyhf[cf]{\footnotesize Page \thepage\ of \pageref*{LastPage}}
%\fancyhf[rf]{}
\renewcommand{\headrule}{} % removes horizontal header line

% Fotter options for first page

\fancypagestyle{firstpagestyle}{
  \renewcommand{\thedate}{\textmd{}} % removes horizontal header line
  \fancyhf{}
  \fancyhf[lh]{\ttfamily M.Sc. Computer Science\\KTH Royal Institute of Technology}
  \fancyhf[rh]{\ttfamily Period 2\\\today}
  \fancyfoot[C]{\footnotesize Page \thepage\ of \pageref*{LastPage}}
  \renewcommand{\headrule}{} % removes horizontal header line
}
%###############################################################################
% Todo: define Title

\title{
  \normalsize{DD2358 VT25 Introduction to}\\
  \normalsize{High Performance Computing}\\
  \large{Assignment X}\\
}
\author{
  \small Author 1\\[-0.75ex]
%  \footnotesize\texttt{MN: }\\[-1ex]
  \scriptsize\texttt{xx@kth.se}
  \and
  \small Author 2\\[-0.75ex]
%  \footnotesize\texttt{MN: }\\[-1ex]
  \scriptsize\texttt{xx@kth.se}
  \and
  \small Author 3\\[-0.75ex]
%  \footnotesize\texttt{MN: }\\[-1ex]
  \scriptsize\texttt{xx@kth.se}
  \and
  \small Paul Mayer\\[-0.75ex]
%  \footnotesize\texttt{MN: }\\[-1ex]
  \scriptsize\texttt{pmayer@kth.se}
}
\date{}

%###############################################################################
% define Commands

\newcommand{\N}{\mathbb{N}}
\newcommand{\R}{\mathbb{R}}
\newcommand{\Z}{\mathbb{Z}}
\newcommand{\I}{\mathbb{I}}

\newcommand{\E}{\mathbb{E}}
\newcommand{\Prob}{\mathbb{P}}

\renewcommand{\epsilon}{\varepsilon}

% Todo: Set Counter to Excercise Sheet Number
%\setcounter{section}{1}
%\setcounter{subsection}{1}

%###############################################################################
%###############################################################################

\begin{document}
\maketitle
\thispagestyle{firstpagestyle}

% \tableofcontents
\listoftodos

\vspace{1em}

%---
%
\section*{Prefix}
\todo[inline]{Make sure title and headers are correctly changed!}
\todo[inline]{Change Author names and mails.}

% content begin
%

\section{Profiling the Julia Set Code}
\subsection{Calculate the Clock Granularity of different Python Timers}
For profiling we used the method given on the exercise page.
On the Apple Silicon M1, I recieved the following output, using the best out of 1000 runs each:
%\texttt{\$ python ClockGranularity.py}\\
%\texttt{time.time: 7.152557373046875e-07 s}\\
%\texttt{time.time\_ns: 7.680000000000001e-07 s}\\
%\texttt{timeit.default\_timer: 8.300412446260452e-08 s}
\begin{lstlisting}[language=bash,basicstyle=\ttfamily]
  $ python ClockGranularity.py
  time.time: 7.152557373046875e-07 s
  time.time_ns: 7.680000000000001e-07 s
  timeit.default_timer: 8.300412446260452e-08 s
\end{lstlisting}

\subsection{Timing the Julia set code functions}
The M1-chip has a clock frequency of 3.2GHz.
\todo[inline]{write anything useful here. I don't see any connection yet?}
\begin{lstlisting}[language=bash,basicstyle=\ttfamily]
  & python JuliaSet.py
  number of runs: 20
  desired width: 1000
  max iterations: 300

  calc_pure_python
      mean: 2.39370298139911 s
      std: 0.019263214648773588 s

  calculate_z_serial_purepython
      mean: 2.534685760299908 s
      std: 0.02066843684620983 s
\end{lstlisting}

\subsection{Profile the Julia set code with cProfile and line\_profiler the computation}
\todo[inline]{1.3: everything}
\subsection{Memory-profile the Juliaset code. Use the memory\_profiler and mprof to profile the computation in JuliaSet code}
\todo[inline]{1.4: everything}

\newpage
\section{Profiling Diffusion Process Code}
\subsection{Profile the diffusion code with cProfile and line\_profiler the computation}
\todo[inline]{2.1: everything}
\subsection{Memory-profile the diffusion code. Use the memory\_profiler and mprof to profile the computation}
\todo[inline]{2.2: everything}

\newpage
\section{Develop your profiler tool for monitoring CPU percentage use with psutil}
\todo[inline]{3.1: everything}

% content end
%###############################################################################

% TODO: bibliograpghy when needed
% \printbibliography

\end{document}
